\documentclass[a4paper]{article}
% load exercise-specific preamble
\usepackage{physics718_ss21}

% Write number of the exercise sheet
\examNum{1}

\begin{document}

\renewcommand\maketitledesign%
{%  \indent
    \makebox[\textwidth]{\normalsize
        \shortstack[l]{\strut\webauthor}\hfill
        \shortstack[r]{\strut\@date \\\webuniversity}}\\
    \makebox[\textwidth]{~}\\
    \makebox[\textwidth]{\large
	\shortstack[l]{\websubject}}%
}
    % This file creates the style for the header
% -------- Header --------------------------------------------------------------
\maketitle              % This command uses the header style created in file eqexam.cfg
\begin{center}
\hrule  \vspace{0.2cm}
\webtitle%
\vspace{0.2cm} \hrule
\end{center}
% ------- Running header --------------------------------------------------------
\rhead{SS 21}        % Text written here appears in the running header right side
\begin{verB}
\chead{-- Seite \arabic{page}\space --}
\end{verB}

\begin{exam}[Presence]{Presence}
  \begin{instructions}[Introduction]

    Welcome to the exercises in \emph{Programming for Physics and Astronomy} in
    the Summer Semester 2021. This year for the first time the programming
    language \emph{Python} will be taught, which due to its simplicity and
    interactivity has arisen as the most popular language for data analysis. In
    the last decades, it has also taken over analysis in high energy physics (HEP). We
    will learn how to apply python to solve problems to data science and HEP in
    practice and in particular learn how to use machine learning via the
    \emph{Keras} library.

    We assume that you have some basic familiarity with programming concepts
    such as variables, functions and control flow (\py{if},
    \py{for}, \ldots) and working in a shell with command line programs.
    We will repeat some basics in the first two exercises, but this can only be
    very limited in scope, so if you learned a different programming language in
    the past or are just rusty, I recommend also going through some of the many
    python exercises on the internet, for example the
    \href{https://swcarpentry.github.io/python-novice-inflammation}{Software
      Carpentry python introduction}\footnote{%
      \url{https://swcarpentry.github.io/python-novice-inflammation}}.
  \end{instructions}

  \begin{instructions}[Get familiar with the development environment]
   Make sure you that you have properly installed the course environment with
   \texttt{conda} and VS~Code according to the instructions
   \href{https://ecampus.uni-bonn.de/goto_ecampus_fold_2146080.html}{%
     on eCampus}\footnote{%
     \url{https://ecampus.uni-bonn.de/goto_ecampus_fold_2146080.html}}.

    \begin{problem*}\textbf{Command line basics and simple control flow}

      Open a terminal and activate the conda environment for the physics718
      exercise via

      \begin{bashcode}
        conda activate physics718
      \end{bashcode}

      Your terminal prompt should now show \texttt{(physics718)} as your active
      python environment. While in it, the \texttt{python} command points
      to the python executable installed by conda instead of your default system
      python.

      \begin{parts}
        \item Type \bash{python} press \keys{\return}. Now you are in a basic
          interactive python interpreter. You can type a line of code, press
          \keys{\return} and the code will be evaluated and the result printed.
          This can be your quick calculator. Find out the python-result for
          $0.1 + 0.2$ in python (remember how computers encode numbers). To exit
          the python execute the \py{exit()} function or type \texttt{Ctrl+d}.
        \item Can you find your python version in the output on the screen? Also
          find how to print the version via a command line argument to
          \texttt{python} (check \bash{python -h}).
          % Bonus: Can you also find out where on the file system the python executable is located?
        \item Instead of the basic python interpreter, for interactive work it
          is more convenient to use Jupyter/IPython, which offers better
          auto-completion and help. Run \bash{jupyter console}. To print 100
          numbers from 0 to 99 you can for example use
          \begin{pythoncode}
            for i in range(100):
                print(i)
          \end{pythoncode}

          Modify the code to only print odd numbers.

          Hint: Use \py{if} and
          the modulo operator \py{%}. Try out how the operator works
          interactively, e.g.\ type \py{9 % 3}.
          Also remember that 0's are treated as boolean \py{False} inside
          conditions while all other numbers are \py{True}, you can validate
          this with e.g.\ \py{print(bool(0))} and \py{print(bool(1))}.

        \item In Jupyter you can find help for all python objects with documentation
          strings via the question mark symbol \texttt{?}. To get a general
          IPython help, just type \texttt{?} end press \keys{\return}.
        \item Find the help for
          the print function via \py{print?}. Add a keyword argument to the \py{print} function to print all
          numbers in a single row.
        \item  When given an argument, the \texttt{python} command can be used to run
          python programs. For example, run \bash{bash zen.py} and be enlightened.
      \end{parts}
    \end{problem*}

    \begin{problem}\textbf{Visual Studio Code}\\
      Interactive environments can be useful for quick calculations and for
      trying things out, but if you want to re-use and share your code, you need
      to organize it into files, which is best done via an editor or integrated
      development environment (IDE). In this course we use VS~Code, which offers
      a python IDE via the python plugin.

      Open \texttt{code} inside the conda environment
    \end{problem}

    \begin{problem}\textbf{Jupyter Notebooks}

    \end{problem}

  \end{instructions}

  \begin{problem}\textbf{Data types}
    Hello World Problem
  \end{problem}
  \begin{problem}\textbf{Assignment}
    Hello World Problem
  \end{problem}
  \begin{problem}\textbf{Control flow}
    Hello World Problem
  \end{problem}
  \begin{problem}\textbf{Function}
    Hello World Problem
  \end{problem}
\end{exam}

\newpage
\begin{exam}[Homework]{Homework}
  \begin{instructions}[\textbf{Homework}]
    Hello World Homework instructions
  \end{instructions}

  \begin{problem}[3]{\textbf{Functions}}
    Homework Problem
  \end{problem}
\end{exam}

\end{document}
