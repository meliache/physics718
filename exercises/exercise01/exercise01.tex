\documentclass[a4paper, draft=False]{scrartcl}
% load exercise-specific preamble
\usepackage{physics718_ss21}

% Write number of the exercise sheet
\examNum{1}

\begin{document}

\renewcommand\maketitledesign%
{%  \indent
    \makebox[\textwidth]{\normalsize
        \shortstack[l]{\strut\webauthor}\hfill
        \shortstack[r]{\strut\@date \\\webuniversity}}\\
    \makebox[\textwidth]{~}\\
    \makebox[\textwidth]{\large
	\shortstack[l]{\websubject}}%
}
    % This file creates the style for the header
% -------- Header --------------------------------------------------------------
\maketitle              % This command uses the header style created in file eqexam.cfg
\begin{center}
\hrule  \vspace{0.2cm}
\webtitle%
\vspace{0.2cm} \hrule
\end{center}
% ------- Running header --------------------------------------------------------
\rhead{SS 21}        % Text written here appears in the running header right side
\begin{verB}
\chead{-- Seite \arabic{page}\space --}
\end{verB}

\begin{exam}[Presence]{Presence}
  \begin{instructions}[Introduction]

    Welcome to the exercises in \emph{Physics718: Programming for Physics and Astronomy} in
    the Summer Semester 2021. This year for the first time the programming
    language \emph{Python} will be taught, which has become the most popular
    language for data analysis in both HEP and the private sector.

    We assume familiarity with basic programming concepts such as variables,
    functions and control flow (\py{if}, \py{for}, \ldots) and working in a
    shell with command line programs. We will repeat some basics in the first
    two exercises, but this can only be very limited in scope, so if you learned
    a different programming language in the past or are just rusty, there is
    plenty of beginner self-study material on the internet which you can work
    through in parallel to the course, for example the
    \href{https://swcarpentry.github.io/python-novice-inflammation}{Python
      Introduction}\footnote{%
      \url{https://swcarpentry.github.io/python-novice-inflammation}} by
    Software Carpentry to name just one.
  \end{instructions}

  \begin{instructions}[Obtaining the exercise repository]

    You can download the code repository that accompanies the exercise sheet
    from eCampus as a \texttt{physics718\_ex01.zip} archive. To unpack it, you can use
    your file manager or the \bash{unzip} command in the linux/Mac terminal.
  \end{instructions}

  \begin{instructions}[Getting familiar with the development environment]

   Make sure you that you have properly installed the course environment with
   \texttt{conda} and VS~Code according to the instructions
   \href{https://ecampus.uni-bonn.de/goto_ecampus_fold_2146080.html}{%
     on eCampus}\footnote{%
     \url{https://ecampus.uni-bonn.de/goto_ecampus_fold_2146080.html}}.

    \begin{problem*}\textbf{Command line basics and simple control flow}

      Open a terminal and activate the conda environment for the physics718
      exercise via

      \begin{bashcode}
        conda activate physics718
      \end{bashcode}

      Your terminal prompt should now show \texttt{(physics718)} as your active
      Python environment. While in it, the \texttt{python} command points to the
      Python executable installed by conda instead of your default system
      Python.

      In your terminal, use \bash{cd} to navigate to the sub-directory
      \directory{physics718/exercises/exercise01}
      \begin{parts}
        \item Type \bash{python} press \keys{\return}. Now you are in a basic
          interactive Python interpreter. You can type a line of code, press
          \keys{\return} and the code will be evaluated and the result printed.
          This can be your quick calculator. Find out the Python result for
          $0.1 + 0.2$ in Python (remember how computers encode numbers). To exit
          the Python execute the \py{exit()} function or type \texttt{Ctrl+d}.
        \item Can you find your Python version in the output on the screen? Also
          find how to print the version via a command line argument to
          \texttt{python} (hint: check \bash{python -h}).
          % Bonus: Can you also find out where on the file system the python executable is located?
        \item Instead of the basic Python interpreter, for interactive work it
          is more convenient to use Jupyter/IPython, which offer better
          auto-completion and help. Run \bash{jupyter console}. To print 100
          numbers from 0 to 99 you can for example use
          \begin{pythoncode}
            for i in range(100):
                print(i)
          \end{pythoncode}

          Modify the code to only print odd numbers.

          Hint: Use \py{if} and
          the modulo operator \py{%}. Try out how the operator works
          interactively, e.g.\ type \py{9 % 3}.
          Also remember that 0's are treated as boolean \py{False} inside
          conditions while all other numbers are \py{True}, you can validate
          this with e.g.\ \py{print(bool(0))} and \py{print(bool(1))}.

        \item In Jupyter you can find help for all Python objects with documentation
          strings via the question mark symbol \texttt{?}. To get a general
          IPython help, just type \texttt{?} end press \keys{\return}. Find one
          other way to access the help of a Python object. To exit the help pager, press \keys{q}.
        \item Find the help for the print function via \py{print?}. Add a
          keyword argument to the \py{print} function to print all numbers in a
          single row.
        \item  When given an argument, the \texttt{python} command can be used to run
          Python programs. Run \bash{bash zen.py} to be enlightened.
      \end{parts}
    \end{problem*}

    \begin{problem*}\textbf{Visual Studio Code}\\
      If you want to re-use and share your code, you need to organize it into
      files, which is best done via an editor or integrated development
      environment (IDE). In this course we use VS~Code, a modern extensible open
      source editor which offers an IDE-like experience via the Python extension.
      Run \bash{code} inside the terminal with the conda environment to open
      VS~code.
      \begin{parts}
      \item Find out how to open a terminal in VS code. In it you can do
        everything we learned so far in the exercise 1. Check that the Python
        version is 3.7 and that you are in a conda environment. If not, you can
        also change your Python interpreter and environment from withing conda
        via \keys{Ctrl+Shift+P} and typing and selecting \menu{Python: Select
          Interpreter}\footnote{\url{https://code.visualstudio.com/docs/python/environments}}.
      \item Open the file \texttt{exercise01\_attendance.py}. Maybe you will be
        prompted to install the Python extension. Open the extension
        menu on the left (\keys{Ctrl+Shift+X}) and if the \emph{Python} extension is
        not installed yet, search for it in the marketplace and install it.
        If you have time, feel free to look for other useful extensions.
      \item Complete the function template for \py{is_odd()}. To test whether
        your implementation is correct, run the tests for the attendance
        exercise. For that you can just execute the file, either in the terminal
        with \bash{python exercise01_attendance.py} or or by clicking the
        \menu{$\triangleright$} button in the top right of VS~Code. Or use the
        \texttt{unittest} module directly by running \bash{python -m unittest}
        (make sure you are in the \texttt{exercise01} sub-directory).
      \item Complete the function \py{sum_of_odds_while()} which calculates the
        sum of odd numbers $1, 3 \ldots N$ until all tests for that function
        pass. Use a \py{while} loop over natural numbers and your \py{is_odd()}
        function. (There is a trivial mathematical way calculate this sum which
        is used in the test but we're doing an exercise here so it please solve
        it the intended way first.)
      \item One more efficient method to generate odd numbers is by providing a
        step argument to the built-in \py{range()} function. Look up the
        documentation for that function in VS-code by hovering your mouse over
        the function name after writing/uncommenting it. Use that to implement
        \py{sum_of_odds_builtins()}.
      \end{parts}
    \end{problem*}

    \begin{problem*}\textbf{Jupyter Notebooks}

      Jupyter Notebooks are interactive programming environments which are great
      for exploratory programming and data-analysis. Instead of entering code in
      line by line as we already did with the normal python interpreter,
      notebooks allow you to organize code in so-called \emph{cells} which can
      contain arbitrary numbers of lines of code and can be executed arbitrary
      order. One advantage is that you can see your results right away and this
      is especially useful when plotting graphs, which we will do later, because
      Jupyter Notebooks allow displaying the plots in-line together with the
      code. They also allow for the creation of text cells (in the \href{%
        https://jupyter-notebook.readthedocs.io/en/latest/examples/Notebook/Working%20With%20Markdown%20Cells.html
      }{\emph{markdown}} format) and thus enable literate programming where code
      and documentation are intermixed. We recommend one of two ways to use
      Jupyter Notebooks:

      \begin{description}
      \item[via Jupyter Lab:] The most widespread method is to use Jupyter from the web-browser
        interface. For that, type the command
        \begin{bashcode}
          jupyter lab
        \end{bashcode}
        into a terminal with the physics718 environment and a browser window
        with the URL being something like \texttt{localhost:8888} should open.
        What happens is that the terminal command creates a Jupyter Server which
        is running at port 8888 on your computer (\texttt{localhost}) and you
        interact with it via the Jupyter Lab client in the web browser. When you
        read the online documentation for Jupyter, it will be mostly targeted at
        users of the web interface, so I think some familiarity is good.
      \item[via VS Code:] VS~Code supports working with directly with Jupyter Notebooks, for
        that you just need to open a notebook file (ending with \texttt{.ipynb})
        with VS~Code, thus you can do everything you need from VS~code without
        requiring you to open your browser. VS~Code still launches a Jupyter
        server in the background, but it abstracts this away.
      \end{description}

      You can try both, but be careful when using them at the same time, as to
      you might overwrite your progress accidentally.

      \textbf{Task:} Open the file \texttt{exercise01\_attendance.ipynb} and
      work through the exercises in the notebook and get familiar with it.
    \end{problem*}
  \end{instructions}
\end{exam}

\newpage
\begin{exam}[Homework]{Homework}
  \begin{instructions}[\textbf{Homework}]
    For the homework you will practice writing some more interesting functions.
    Write the solutions to the following tasks by filling in the missing code in
    \texttt{exercise01\_homework.py}. Remember to run the tests by running the file
  \end{instructions}

  \begin{problem}[4]\textbf{Odd Fibonacci numbers}
    Write a function \py{odd_fibonacci_numbers(n_numbers)} that returns a
    list of \py{n_numbers} odd Fibonacci
    numbers\footnote{\url{https://en.wikipedia.org/wiki/Fibonacci_number}},
    starting with $1, 1, 3, 5, \ldots$. Don't rely on external packages.
  \end{problem}
  \begin{problem*}[\auto]\textbf{From sum of squares to variance}
    \begin{parts}
     \item\PTs{2} Write a function \py{sum_square_difference} that takes a list
       of numbers and returns the difference between the sum of the squares and
       the square of the sum $\sum_i x_i^2 - \left(\sum_i x_i\right)^2$.
     \item\PTs{2} In python functions are first-class object, that means
       that functions can take other function objects as arguments (just omit
       the parentheses). Use that to write a more general version of
       \py{sum_square_difference} by implementing the higher order function
       \py{func_square_difference} which accepts an arbitrary aggregating
       function $f$ and returns $f(x_1^2\ldots x_N^2) - f(x_1\ldots x_N)^2$.
     \item\PTs{2} The variance of a collection can be calculated obtained with
       \begin{equation*}
       \mathrm{Var}(\vec{x}) = \mathrm{mean}(x_1^2\ldots x_N^2) - \mathrm{mean}(x_1\ldots x_N)^2\label{eq:1}
     \end{equation*}
     Implement the \py{mean} function. Use that as an argument to the previously defined
     \py{func_square_difference} function to calculate and return the variance
     in the \py{variance} function.
    \end{parts}
  \end{problem*}
\end{exam}

\end{document}
