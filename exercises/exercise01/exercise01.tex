\documentclass[english,ngerman]{article}
% load exercise-specific preamble
\usepackage{physics718_ss21}

% Select the exercise group
\forVersion{e}
% Write number of the exercise sheet
\examNum{1}

\begin{document}

\renewcommand\maketitledesign%
{%  \indent
    \makebox[\textwidth]{\normalsize
        \shortstack[l]{\strut\webauthor}\hfill
        \shortstack[r]{\strut\@date \\\webuniversity}}\\
    \makebox[\textwidth]{~}\\
    \makebox[\textwidth]{\large
	\shortstack[l]{\websubject}}%
}
    % This file creates the style for the header
% -------- Header --------------------------------------------------------------
\maketitle              % This command uses the header style created in file eqexam.cfg
\begin{center}
\hrule  \vspace{0.2cm}
\webtitle%
\vspace{0.2cm} \hrule
\end{center}
% ------- Running header --------------------------------------------------------
\rhead{SS 21}        % Text written here appears in the running header right side
\begin{verB}
\chead{-- Seite \arabic{page}\space --}
\end{verB}

\begin{exam}[Presence]{Presence}
  \begin{instructions}[Introduction]

    Welcome to the exercises in \emph{Programming for Physics and Astronomy} in
    the Summer Semester 2021. This year for the first time the programming
    language \emph{Python} will be taught, which due to its simplicity and
    interactivity has arisen as the most popular language for data analysis. In
    the last decades, it has also taken over analysis in high energy physics (HEP). We
    will learn how to apply python to solve problems to data science and HEP in
    practice and in particular learn how to use machine learning via the
    \emph{Keras} library.

    We assume that you have some basic familiarity with programming concepts
    such as variables, functions and control flow (\py{if},
    \py{for}, \ldots) and working in a shell with command line programs.
    We will repeat some basics in the first two exercises, but this can only be
    very limited in scope, so if you learned a different programming language in
    the past or are just rusty, I recommend also going through some of the many
    python exercises on the internet, for example the \emph{Software Carpentry}
    \href{https://swcarpentry.github.io/python-novice-inflammation}{Python Introduction}\footnote{%
      \url{https://swcarpentry.github.io/python-novice-inflammation}}.
    I will collect other material for self-study on eCampus.

  \end{instructions}
  \begin{problem}\textbf{Data types}
    Hello World Problem
  \end{problem}
  \begin{problem}\textbf{Assignment}
    Hello World Problem
  \end{problem}
  \begin{problem}\textbf{Control flow}
    Hello World Problem
  \end{problem}
  \begin{problem}\textbf{Function}
    Hello World Problem
  \end{problem}
\end{exam}

\newpage
\begin{exam}[Homework]{Homework}
  \begin{instructions}[\textbf{Homework}]
    Hello World Homework instructions
  \end{instructions}

  \begin{problem}[3]{\textbf{Functions}}
    Homework Problem
  \end{problem}
\end{exam}

\end{document}
