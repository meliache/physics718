\documentclass[a4paper]{article}
% load exercise-specific preamble
\usepackage{physics718_ss21}

% Select the exercise group
\forVersion{e}
% Write number of the exercise sheet
\examNum{1}

\begin{document}

\renewcommand\maketitledesign
{%
    \makebox[\textwidth]{\normalsize
        \shortstack[l]{\strut\webauthor}\hfill
        \shortstack[r]{\strut\\\@date\\\webuniversity}}\\
    \makebox[\textwidth]{~}\\
    \makebox[\textwidth]{\large
	\shortstack[l]{\websubject}}%
}    % This file creates the style for the header
% -------- Header --------------------------------------------------------------
\maketitle              % This command uses the header style created in file eqexam.cfg
\begin{center}
\hrule  \vspace{0.2cm}
\webtitle%
\vspace{0.2cm} \hrule
\end{center}
% ------- Running header --------------------------------------------------------
\rhead{SS 21}        % Text written here appears in the running header right side
\begin{verB}
\chead{-- Seite \arabic{page}\space --}
\end{verB}

\begin{exam}[Presence]{Presence}
  \begin{instructions}[Introduction]

    Welcome to the exercises in \emph{Programming for Physics and Astronomy} in
    the Summer Semester 2021. This year for the first time the programming
    language \emph{Python} will be taught, which due to its simplicity and
    interactivity has arisen as the most popular language for data analysis. In
    the last decades, it has also taken over analysis in high energy physics (HEP). We
    will learn how to apply python to solve problems to data science and HEP in
    practice and in particular learn how to use machine learning via the
    \emph{Keras} library.

    We assume that you have some basic familiarity with programming concepts
    such as variables, functions and control flow (\py{if},
    \py{for}, \ldots) and working in a shell with command line programs.
    We will repeat some basics in the first two exercises, but this can only be
    very limited in scope, so if you learned a different programming language in
    the past or are just rusty, I recommend also going through some of the many
    python exercises on the internet, for example the
    \href{https://swcarpentry.github.io/python-novice-inflammation}{Software
      Carpentry python introduction}\footnote{%
      \url{https://swcarpentry.github.io/python-novice-inflammation}}.
  \end{instructions}

  \begin{instructions}[Get familiar with the development environment]
   Make sure you that you have properly installed the course environment with
   \texttt{conda} and VS~Code according to the instruction
   \href{https://ecampus.uni-bonn.de/goto_ecampus_fold_2146080.html}{%
     on eCampus}\footnote{%
     \url{https://ecampus.uni-bonn.de/goto_ecampus_fold_2146080.html}}.

    \begin{problem*}\textbf{Python command line basics}

      Open a terminal and activate the conda environment for the physics718
      exercise via

      \begin{bashcode}
        conda activate physics718
      \end{bashcode}

      Your terminal prompt should now show \texttt{(physics718)} as your active
      python environment. While in it, the \texttt{python} command points
      to the python executable installed by conda instead of your default system
      python. To deactivate the environment, use \bash{conda deactivate}.

      \begin{parts}
        \item Find the python version that your are using on the command line.
          Bonus: Can you also find out where on the file system the python executable is located?
        \item Type \bash{python} press \texttt{Enter}. Now you should be in an
          interactive python interpreter. You can type a line of code, press
          \texttt{Enter} and the code will be evaluated and the result printed.
          This is will be your quick calculator. To test it, find out the python-result for
          $0.1 + 0.2$ in python. Do you have an idea what might cause the
          result?
        \item Exit the python interpreter by executing the \py{exit()} function or typing
          \texttt{Ctrl+d}. When given an argument, the python command can be used to run
          python programs. Execute for example the file \texttt{zen.py} and be enlightened.
        \item Instead of the basic python interpreter, for interactive work it
          is better to use Jupyter, which offers auto-completion and other
          convenience functions. Run \bash{jupyter console}
          and import\footnote{Python imports are magic. Try \py{import
              antigravity}}. NumPy via \py{import numpy as np}.

          To find out what NumPy is and how to get more information type \texttt{np?}
          In Jupyter the question-mark  \texttt{?} shows the
           of help any object (e.g.\ functions, classes, \ldots). What other method can be used to access the help?
        \item Use the \py{np.arange} and \py{np.sum} functions to find the sum of all odd numbers between 0
          and 200. Use the help of the functions for guidance and use no other functions or operators.
      \end{parts}
    \end{problem*}

  \end{instructions}
  \begin{problem}\textbf{Data types}
    Hello World Problem
  \end{problem}
  \begin{problem}\textbf{Assignment}
    Hello World Problem
  \end{problem}
  \begin{problem}\textbf{Control flow}
    Hello World Problem
  \end{problem}
  \begin{problem}\textbf{Function}
    Hello World Problem
  \end{problem}
\end{exam}

\newpage
\begin{exam}[Homework]{Homework}
  \begin{instructions}[\textbf{Homework}]
    Hello World Homework instructions
  \end{instructions}

  \begin{problem}[3]{\textbf{Functions}}
    Homework Problem
  \end{problem}
\end{exam}

\end{document}
