\documentclass[a4paper, draft=False]{scrartcl}
% load exercise-specific preamble
\usepackage{physics718_ss21}
\usepackage{shapepar}
% Write number of the exercise sheet
\begin{document}

\renewcommand\maketitledesign%
{%  \indent
    \makebox[\textwidth]{\normalsize
        \shortstack[l]{\strut\webauthor}\hfill
        \shortstack[r]{\strut\@date \\\webuniversity}}\\
    \makebox[\textwidth]{~}\\
    \makebox[\textwidth]{\large
	\shortstack[l]{\websubject}}%
}
    % This file creates the style for the header
% -------- Header --------------------------------------------------------------
\maketitle              % This command uses the header style created in file eqexam.cfg
\begin{center}
\hrule  \vspace{0.2cm}
\webtitle%
\vspace{0.2cm} \hrule
\end{center}
% ------- Running header --------------------------------------------------------
\rhead{SS 21}        % Text written here appears in the running header right side
\begin{verB}
\chead{-- Seite \arabic{page}\space --}
\end{verB}


\begin{exam}[Presence]{Presence}
  \begin{instructions}[Presence exercises]

    \begin{problem*}\textbf{Simple class}

      In their most simple form, python classes can be used to store data in
      objects (instances of the class). The data is stored in attributes
      (member variables, denoted with \py{self.<varname>} where \py{self} is
      stands for the instance of the class). Storing data in an object allows
      \emph{encapsulating} it together with functions that operate on that data
      (also sometimes called \emph{state} of the object) or modify it.

      Unpack the latest exercise directory and navigate to
      \directory{physics718/exercises/exercise02}. Open the notebook
      \texttt{exercise02\_attendance.ipynb} and find the beginning of the definition
      for the \py{Person} class. Implement the class in the order of the tasks
      below. Feel free to test your class and different methods interactively in
      the notebook. For your convenience a test is also included.

      \begin{parts}
      \item Implement the \py{__init__} constructor method of the \py{Person} class, so
        that a person object can be instantiated with a first name, last
        name and age. Set the class attributes \py{first_name}, \py{last_name}
        and \py{age}.
        \begin{pythoncode}
          # `person1` is object of the class/type `Person`
          person1 = Person("Max", "Power", 38)
        \end{pythoncode}
      \item Make sure that the attributes of your object are properly set.
        Print the \py{first_name}, \py{last_name} and \py{age} of a person
        object, e.g. \py{print(person.first_name)}
      \item Use the attributes to change the state of an already instantiated
        \py{Person} object.
      \item Implement the method (member function) \py{full_name},
        which returns a string with the full name of the person, i.e.
        \py{"<first name> <last name>"}.
      \item Use the \py{full_name} function to implement the \py{__str__}
        method, which by convention returns a readable string representation
        of the class, which is called when the object is converted to a string
        (e.g.\ implicitly when printing). Make sure your implementation works
        by printing the person object.
        \begin{pythoncode}
          person2 = Person("Margaret", "Hamilton", 84)
          print(person2)  # Margaret Hamilton
        \end{pythoncode}
      \item Also implement the \py{say_hello} method which prints \texttt{Hello <first name> <last name>}.
      \item Let's make use of the age information. Implement the method
        \py{older_than(self, other)}, that takes one other person and returns \py{True}
        if the current person is older than the other person. Check that
        your function works, e.g.\
        \begin{pythoncode}
          person1.older_than(person2)  # False
        \end{pythoncode}
      \item Add  syntactic sugar to be able to compare the age of
        persons with the greater-than \py{>} operator, e.g.\
        \begin{pythoncode}
          person2 > person1  # True
        \end{pythoncode}
        For that purpose, implement the \py{__gt__} method (e.g. set it to be equal
        to the \py{older_than} method).

        \notebox{In reality one wouldn't do this, because we might
          add properties like \py{size}, \py{weight} etc.\ later and
          then the simple comparison operators are ambivalent.}
      \end{parts}
    \end{problem*}
    \begin{problem*}\textbf{Extend \texttt{Vector2D}}

      Go in the notebook to section \emph{2. Extending a class}. Read the text,
      instantiate some vectors and do some calculations with them to get
      familiar. Then open the file \texttt{vector\_2d.py} with VS Code to extend
      it.

      \begin{parts}
      \item Implement the \py{/} operator via the \py{__truediv__} method for
        division with a scalar.
      \item Implement a \py{normal} method which returns a normalized vector.
      \item Implement a method \py{angleTo} to get the angle between two vectors.
      \end{parts}
    \end{problem*}

    \begin{problem*}\textbf{Particle via inheritance}

      Imagine that we want to describe a particle by its velocity vector and its
      mass, but are not interested in its position. Then, we can implement a
      particle as a sub-class of \py{Vector2D}. Go to section 3 in the notebook
      and implement the \py{Particle} class.

        Mind that the x and y should represent the particle velocity for the
        purpose of this problem.

    \begin{parts}
      \item Implement an \py{__init__(self, vx, vy, mass)} method that sets the
        x/y-components of the super class and sets the mass property
        (hint: use \py{super().__init__}).
      \item Implement the \py{momentum()} and \py{energy()} methods. Note that
        \py{momentum()} should return an instance of \py{Vector2D}
    \end{parts}
    \end{problem*}
  \end{instructions}
\end{exam}

\begin{exam}[Homework]{Homework}
  \begin{instructions}[\textbf{Homework}]
    \begin{problem*}[\auto]\textbf{Particle via class composition}

      Properties of a class can themselves be instances of complex classes. This
      is known as composition. This can be described as a \emph{has a}
      relationship in contrast to the \emph{is a} relationship of inheritance.

      We can use that to implement a new \py{Particle} class which is aware of both
      its momentum and position, which can be stored as \py{Vector2D} properties.

      Open the file \texttt{exercise02\_homework.py} to implement it.

      \begin{parts}
        \item\PTs{2} Implement the particle constructor
          \begin{pythoncode}
            __init__(self, position, momentum, mass)
          \end{pythoncode}
          where \py{position} and \py{momentum} are
          instances of \py{Vector2D}.
          Set the class properties of the same names as the arguments.
        \item\PTs{2} Implement the method \py{get_velocity()} which returns a
          \py{Vector2D} with the particle velocity calculated from the momentum
          and mass.
        \item\PTs{2} Implement the method \py{get_energy()} which returns the
          kinetic energy of the particle.
          \py{Vector2D} with the particle velocity calculated from the momentum
          and mass.
         \item\PTs{2} We can easily obtain the momentum and mass of a particle
           easily via properties, e.g.\
           \begin{pythoncode}
             p = Particle(Vector2D(0, 0), Vector2D(2, 1), 1)
             print(p.momentum)  # Vector2D(2, 1)
             print(p.mass)  # 1
           \end{pythoncode}
           For the velocity and energy however, we need to call methods (with
           parentheses), e.g. \py{p.get_velocity()}. Python allows adding a
           so-called ``\py{@property}-decorator'' to the top of methods, which
           tells python to treat the function below as a property and allows you to
           call it without the parentheses, e.g. \py{print(particle.velocity)}.

           Finish the implementation of the \py{velocity} property method and
           create likewise an \py{energy} property method. Both should
           return the return values of the previously defined getter functions.
      \end{parts}
    \end{problem*}

\begin{problem*}[\auto]\textbf{Complex number class with operators}

  Python has support for complex numbers built in (try typing \py{c = 1 + 2j}),
  but we will try implementing our own \py{Complex} number class. This time there is
  no template and you have to write the class mostly from scratch, but now you
  have other classes (e.g. \py{Vector2D}) to give you ideas. Don't use python's
  complex numbers internally.
  \begin{parts}
  \item\PTs{1} Write the class \py{Complex} with an \py{__init__} method
    that take real and imaginary component and sets the properties \py{self.real} and
    \py{self.imag}.
   \item\PTs{2} Implement the operator methods for equality comparison
     (\py{==}), for the addition and subtraction (\py{+}, \py{-}) of complex numbers .
   \item\PTs{2} Implement the methods \py{magnitude}, \py{conjugate} and
     \py{to_polar} to get the magnitude, complex conjugate and polar
     representation ($r$, $\phi$) of a complex number, respectively.
   \item\PTs{3} Implement the multiplication of two complex numbers with the
     operator \py{*}. Allow one of the two numbers to be a scalar number.
  \end{parts}
  You can implement other methods if you like, but they are not required. For
  example for interactive debugging it might help to implement the \py{__str__}
  and \py{__repr__} methods.
\end{problem*}


\notebox{If you liked the interactive experimentation in the Jupyter notebook
  during the attendance exercise, feel free to also prototype your homework
  solutions in a notebook and just paste/import them back into the homework
  submission file \texttt{exercise02\_homework.py} once you are done. Make sure
  that at least the tests pass, which are minimal a requirement for getting
  the points.}
  \end{instructions}

\end{exam}

\end{document}
